%% Beginning code for all standard physics latex documents

%Created on: May 8, 2014    Edited by: Wesley Kyle
%Edited on:	May 12, 2016	Edited by: P. Gimby - cleaned up the code to remove unneeded packages
%Edited on:	May 13, 2016	Edited by: P. Gimby - collected a few more packages used in 325.
%Edited on:	May 16, 2016	Edited by: P. Gimby - fixed page numbering error.
%Edited on: May 20, 2016	Edited by: Alex Shook - Added packages for 497

\documentclass[justified]{tufte-book}
\usepackage{graphicx} % allow embedded images
\setkeys{Gin}{width=\linewidth,totalheight=\textheight,keepaspectratio}
\usepackage{amsmath}  % extended mathematics
\usepackage{bm}  % bold font in math mode
\usepackage{longtable} %lets long tables flow into multiple pages instead of running off the page or having to break tables up manually
\usepackage{booktabs} % book-quality tables
\usepackage{units}    % non-stacked fractions and better unit spacing
\usepackage{multicol} % multiple column layout facilities
\usepackage{tikz} %for drawing nice pictures
\usepackage{indentfirst} % makes first line of each new section be indented
\usepackage{enumitem} % extended options for the enumerate environment
\usepackage{soul} % gives more typestting options like spacing, underline, and strike-through
\usepackage{marvosym} %extra symbols package
\usepackage{multirow} % for special table controls
\usepackage[singlelinecheck=false]{caption} % allow captions w/o figure number
\captionsetup{compatibility=false} % corrects in issue with the caption package
\usepackage{float} % allows for contorl over position of figures and tables
\allowdisplaybreaks % allows equations to span two pages if needed
\usepackage{mathrsfs} % fancy math symbols
\usepackage{multirow} % for special table controls
\usetikzlibrary{arrows,shapes,snakes,calc,patterns,3d} % addon to tikz
\usetikzlibrary{circuits.ee.IEC} % addon to tikz
\usepackage{pgfplots} % package for making plots of functions
\usepackage{gensymb} % symbols i,e. degrees
\usetikzlibrary{decorations.pathmorphing} % to draw the springs
\tikzset{circuit declare symbol = ac source}
\tikzset{set ac source graphic = ac source IEC graphic}
\usepackage{changepage} % allows for full page environment
\usepackage{comment} % allows comment tags for large sections

% define new page style that puts page numbers in the middle
%\begin{comment}
\fancypagestyle{custom}{
\fancyhf{} % clear all header and footer fields
\fancyheadoffset{0pt}
\fancyfootoffset{0pt}
\fancyfoot[C]{\thepage}
\renewcommand{\headrulewidth}{0pt}
\renewcommand{\footrulewidth}{0pt}}
\pagestyle{custom}
%\end{comment}

%below creates a new circuit symbol for AC sources
\tikzset{
         ac source IEC graphic/.style=
          {
           transform shape,
           circuit symbol lines,
           circuit symbol size = width 3 height 3,
           shape=generic circle IEC,
           /pgf/generic circle IEC/before background=
            {
             \pgftransformresetnontranslations
             \pgfpathmoveto{\pgfpoint{-0.8\tikzcircuitssizeunit}{0\tikzcircuitssizeunit}}
             \pgfpathsine{\pgfpoint{0.4\tikzcircuitssizeunit}{0.4\tikzcircuitssizeunit}}
             \pgfpathcosine{\pgfpoint{0.4\tikzcircuitssizeunit}{-0.4\tikzcircuitssizeunit}}
             \pgfpathsine{\pgfpoint{0.4\tikzcircuitssizeunit}{-0.4\tikzcircuitssizeunit}}
             \pgfpathcosine{\pgfpoint{0.4\tikzcircuitssizeunit}{0.4\tikzcircuitssizeunit}}
             \pgfusepathqstroke
            }
          }
        }
% end of circuit symbol
%\begin{document}
%%%end individual beginning code/,$d


%  \begin{titlepage}
%    \vspace*{\fill}
%    \begin{center}
%      \huge{{\bf TITLE1}}\\[0.4cm]
%      \huge{TITLE2}\\[0.4cm]
%      \LARGE{Laboratory Manual}\\[0.4cm]
%      \large{SEASON YEAR}
%    \end{center}
%    \vspace*{\fill}
%  \end{titlepage}
%\maketitle

%\begin{spacing}{0.5}
%\tableofcontents
%\end{spacing}

%NEW PHYS 497 PACKAGES AND COMMANDS

%Subcaption package: Allows subfigures to be placed side by side, and labeled with individual captions (Added June 1, 2016)
\usepackage{subcaption}

%Array package: Allows for addiation specifications in arrays (Added May 6, 2016)
\usepackage{array}

%newcolumntype: Allows one to specify a fixed column width (Added May 6, 2016)
\newcolumntype{L}[1]{>{\raggedright\let\newline\\\arraybackslash\hspace{0pt}}m{#1}}
\newcolumntype{C}[1]{>{\centering\let\newline\\\arraybackslash\hspace{0pt}}m{#1}}
\newcolumntype{R}[1]{>{\raggedleft\let\newline\\\arraybackslash\hspace{0pt}}m{#1}}

%circuits.logic.US, circuits.logic.IEC: For drawing logic gates in Tikz (Added May 6, 2016) 
\usetikzlibrary{circuits.logic.US,circuits.logic.IEC}

\newcommand{\PGT}{ %PGT: positive going transition
\begin{tikzpicture}
\draw[-angle 60] (0,0) -- (0,5pt);
\draw (0,5pt) -- (0,6pt) -- (5pt,6pt);
\draw (-5pt,0) -- (0,0);
\end{tikzpicture}
}





%TEST
\usepackage{geometry}
\pagestyle{fancy}

%\usepackage[caption=false]{subfig}

%\makeatletter
%\renewenvironment{figure}[1][htbp]{%
%  \@tufte@orig@float{figure}[#1]%
%}{%
%  \@tufte@orig@endfloat
%}

%\renewenvironment{table}[1][htbp]{%
%  \@tufte@orig@float{table}[#1]%
%}{%
%  \@tufte@orig@endfloat
%}
%\makeatother

% use instead of subfigure
\makeatletter
\newenvironment{multifigure}[1][htbp]{%
  \@tufte@orig@float{figure}[#1]%
}{%
  \@tufte@orig@endfloat
}
\makeatother

\makeatletter
\newenvironment{mainfigure}[1][htbp]{%
\@tufte@orig@float{figure}[#1]
\begin{adjustwidth}{}{-153pt}}
{\end{adjustwidth}\@tufte@orig@endfloat}%
\makeatother

\makeatletter
\newenvironment{maintable}[1][htbp]{%
\@tufte@orig@float{table}[#1]
\begin{adjustwidth}{}{-153pt}}
{\end{adjustwidth}\@tufte@orig@endfloat}%
\makeatother

%%%% Labatorial Cross-over labs need this code. This should be temporary PG Dec 7, 2016

\newcounter{questioncounter}
\setcounter{questioncounter}{0}
\newcounter{checkpointcounter}
\setcounter{checkpointcounter}{0}
\newcounter{figurecounter}
\setcounter{figurecounter}{0}
%%%%%%%%%%%%%%%%%%%%%%%%%%%%%%%%%%%%%%%%%%%%%%%%%%%%%%%

\newcommand{\checkpoint}{
 \fbox{\begin{minipage}{0.2\textwidth}
 %\includegraphics[width=0.5\textwidth]{stop}
 \end{minipage}
 \begin{minipage}{1.0\textwidth}
 {\bf CHECKPOINT \addtocounter{checkpointcounter}{1} \arabic{checkpointcounter}: Before moving on to the next part, have your TA check the results you obtained so far.}
 \end{minipage}}}

%%% end labatorial cross-over code.

% New environment for placing figure captions under the figure
%\makeatletter
%\newenvironment{mainfigure}{\textwidth}[1][htbp]{%
%\@tufte@orig@float{figure}[#1]%
%}{%
%\@tufte@orig@endfloat
%}
%\makeatother

\documentclass[justified]{book}
\usepackage{dirtree}
\usepackage{listings}
\usepackage{xcolor}
\definecolor{light-gray}{gray}{0.9}
\begin{document}



\title{Educational Laboratory Website Manual}
\maketitle
\tableofcontents
%\vspace{-1.5cm}
%\today
%\begin{adjustwidth}{}{-153pt}

\section{Introduction}





\section{Making Changes to Website}

All changes to the website should be made on the development space on slug (/usr/local/master/pjl-web). The only exception to this rule is that the equipment database equipmentDB.xml can be modified live by using the inventory website in edit mode. Because of this it is important to sync the development and live version of the website before making changes.

\subsection{Sync Live Version with Development Space}

\begin{enumerate}
\item Sync web-server to github using the procedure outlined in section \ref{proc:sync}. Note that for the web server "watt" the /path/to/web-folder is /var/www/html.

\item Sync development space to github using the procedure outlined in section \ref{proc:sync}. Note that for the development version on "slug" the path is /usr/local/master/pjl-web.

\item Sync the web server from the github using
\begin{itemize}
	\item admin-user@computer /var/www/html\\
	\$ sudo git pull origin master
\end{itemize}

\item Development space is now ready for editing.
\end{enumerate}

\subsection{Order of Making Changes}
\begin{enumerate}
\item Sync live version and development space
\item Changes Inventory
\item Repository
\item Update live version on web server

\end{enumerate}

\subsection{Procedure for updating PJL github}\label{proc:sync}


\begin{enumerate}
\item admin-user@computer /path/to/web-folder\\
	\$ sudo git add .
\item admin-user@computer /path/to/web-folder\\
	\$ sudo git commit -m upload latest ``equimpnetDB.xml''
\item admin-user@computer /path/to/web-folder\\
	\$ sudo git pull origin master
\item admin-user@computer /path/to/web-folder\\
	\$ sudo git push origin master\\
	Username for `https://github.com': admin-user\\
	Password for `https://admin-user@github.com': ********
\end{enumerate}
%\item Upload made to Slug to github.


\section{Repository}

\subsection{Directory structure}

At top most level is a folder called ``repository" that contains all experiment related documents.\\

\noindent At the second level all the files are organized by lab experiment. Each experiment has a folder that is labeled with a naming scheme where the first four characters are the unique identifier number, followed by the name of the lab. The lab name should be descriptive of the experiment itself. In this folder is also a folder called ``Support\_Docs" that contains any documents useful for the experiment, but not actually used to generate the student document. \\

\noindent At the third level files are organized into versions. Each folder follows a naming scheme where the the first four characters are the unique lab identifier number, followed by ``PHYS" followed by the Course Number followed by a two character semester identifier, followed by the year. Each folder contains all the file used to generate the pdf given to students in the course, semester, and year as identified in the folder label. \\

\noindent {\bf Directory structure sample.}

\dirtree{%
.1 /repository.
.2 0072-Nuclear-Decay.
.3 0001-PHYS123FA2017.
.4 lab.tex.
.4 photo.jpg.
.4 student.pdf.
.2 Support\_Docs.
}
\vspace{12pt}
{\bf Documents can only be added to the repository if they meet the following criteria.}
\begin{enumerate}
\item The files include the pdf given to students to be used in their course work.
\item All files need to generate the pdf are included.
\end{enumerate}


%\end{enumerate}
%\item Upload changes made to Watt to github.

\subsection{Adding a new lab to the repository}

Before beginning ensure that all equipment used in the new experiment are in the lab inventory, and have equipment ID number. 

\begin{enumerate}
\item Create a folder for the new lab (example ``new-lab-folder"), and place all files for generating student pdf, and the student pdf in new-lab-folder
\item Inside new-lab-folder make a directory called ``Support\_Docs", and put all documents relevant to lab, but not needed for generation of pdf into it. This might include research papers, sample data, Excel spreadsheets, etc.
\item Inside new-lab-folder create a file called info.csv and add all relevant information to the file.\\

%Example: The file info.csv needs to be in this form. See below for more details.

	\begin{lstlisting}[backgroundcolor = \color{light-gray},
					caption = The file info.csv needs to be in this form. See below for more details
                   %language = C,
                   xleftmargin = 2cm,
                   framexleftmargin = 1em
                   ]
Name,Vectors-in-One-and-Two-Dimensions
Type,Labatorial
Disciplines,Newtonian Mechanics,Optics
Topics,Collisions,Momentum
Semester,Fall
Year,2017
Course,325
Equipment,0004-(1)-[0002],0050-(4)
Software,ImageJ,Canon
PDF,title.pdf
\end{lstlisting}


Any text that is \textbf{Bold} must be configured for each lab. text in \textit{italic} are notes that must be removed. Standard text must be left as is.

\begin{itemize}
\item Name, \textbf{Name as to be Seen on Website} \textit{ Use Standard Title Capitalize Convention.}
\item Type, \textbf{Type} \textit{ Must be either Lab or Labatorial.}
\item Disciplines, \textbf{Discipline1, Discipline2} \textit{ Disciplines must comma separated be taken from the approved list} \textbf{Need location of this list.}
\item Topics, \textbf{Topic1, Topic2} \textit{ Topics must comma separated be taken from the approved list} \textbf{Need location of this list.}
\item Semester, \textbf{Semester} \textit{ Winter, Spring, Summer, or Fall}
\item Year, \textbf{Year} \textit{: Four digits.}
\item Course, \textbf{Course Number} \textit{ Three digit number corresponding to the course the experiment was used in.}
\item Equipment, \textbf{equipID-(Amount)-[alternate equipID], equipID-(Amount)} \textit{ equipID is four digit code of equipment in inventory, Amount is how many are needed, alternate ID is the four digit code of equipment in inventory that can be used if the primary unit is not available. IDs amounts and alternate IDs separated by ``-", and items in equipment list separated by ``,"}
\item Software, \textbf{Software1, Software2} \textit{ Name of all software needed. Must be software from the list of supported software} \textbf{Need location of this list}
\item PDF, \textbf{PDF exact Name} \textit{ This needs to be the exact name of the student pdf}
\end{itemize}	

\item Run ``addNewLab.py /path/to/new-lab-folder" from development space. 
\end{enumerate}

\subsection{Repository xml template}

\lstinputlisting[language=xml, linerange = {7-49}, backgroundcolor = \color{light-gray}, basicstyle = \footnotesize, xleftmargin = 2cm, framexleftmargin = 1em]{/usr/local/master/pjl-web/doc/website-manual/code-samples/labDB-template.txt}




\section{Inventory}

\subsection{Inventory structure}

Each item in the inventory should have a unique identifier number, and a unique name. Each item has a place for any number of manuals, and one picture. 


\subsection{Adding New Equipment}



\begin{enumerate}
\item Compile all rel

	\begin{lstlisting}[backgroundcolor = \color{light-gray}, 	caption = The file info.csv needs to be in this form.
	               language = csv, xleftmargin = -2cm, xrightmargin = -2cm,
                   framexleftmargin = 1em
                   ]
                   
name; Unique Name of Kit, is_kit; False, kit ; item1 ; item2 ; item3 (2)
name; Unique Name of Equipment

\end{lstlisting}

\end{enumerate}

\subsection{Adding photos of equipment}




\section{Scripts}

\subsection{Add New Lab - addNewLab.py}
%	\begin{lstlisting}[backgroundcolor = \color{lightgray},
 %                  language = python,
  %                 xleftmargin = -4cm,
   %                framexleftmargin = 1em,
    %               xrightmargin = -4cm,
     %             framexrightmargin = 1em
      %             ]


\lstinputlisting[language=python, backgroundcolor = \color{light-gray}, basicstyle = \footnotesize, xleftmargin = -3cm, xrightmargin = -3cm]{/usr/local/master/pjl-web/dev/python-tools/addNewLab.py}

%\end{lstlisting}

%\end{adjustwidth}
\end{document}
