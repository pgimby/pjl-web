% Beginning code for all standard physics latex documents

%Created on: May 8, 2014    Edited by: Wesley Kyle
%Edited on:	May 12, 2016	Edited by: P. Gimby - cleaned up the code to remove unneeded packages
%Edited on:	May 13, 2016	Edited by: P. Gimby - collected a few more packages used in 325.
%Edited on:	May 16, 2016	Edited by: P. Gimby - fixed page numbering error.
%Edited on: May 20, 2016	Edited by: Alex Shook - Added packages for 497

\documentclass[justified]{tufte-book}
\usepackage{graphicx} % allow embedded images
\setkeys{Gin}{width=\linewidth,totalheight=\textheight,keepaspectratio}
\usepackage{amsmath}  % extended mathematics
\usepackage{bm}  % bold font in math mode
\usepackage{longtable} %lets long tables flow into multiple pages instead of running off the page or having to break tables up manually
\usepackage{booktabs} % book-quality tables
\usepackage{units}    % non-stacked fractions and better unit spacing
\usepackage{multicol} % multiple column layout facilities
\usepackage{tikz} %for drawing nice pictures
\usepackage{indentfirst} % makes first line of each new section be indented
\usepackage{enumitem} % extended options for the enumerate environment
\usepackage{soul} % gives more typestting options like spacing, underline, and strike-through
\usepackage{marvosym} %extra symbols package
\usepackage{multirow} % for special table controls
\usepackage[singlelinecheck=false]{caption} % allow captions w/o figure number
\captionsetup{compatibility=false} % corrects in issue with the caption package
\usepackage{float} % allows for contorl over position of figures and tables
\allowdisplaybreaks % allows equations to span two pages if needed
\usepackage{mathrsfs} % fancy math symbols
\usepackage{multirow} % for special table controls
\usetikzlibrary{arrows,shapes,snakes,calc,patterns,3d} % addon to tikz
\usetikzlibrary{circuits.ee.IEC} % addon to tikz
\usepackage{pgfplots} % package for making plots of functions
\usepackage{gensymb} % symbols i,e. degrees
\usetikzlibrary{decorations.pathmorphing} % to draw the springs
\tikzset{circuit declare symbol = ac source}
\tikzset{set ac source graphic = ac source IEC graphic}
\usepackage{changepage} % allows for full page environment
\usepackage{comment} % allows comment tags for large sections

% define new page style that puts page numbers in the middle
%\begin{comment}
\fancypagestyle{custom}{
\fancyhf{} % clear all header and footer fields
\fancyheadoffset{0pt}
\fancyfootoffset{0pt}
\fancyfoot[C]{\thepage}
\renewcommand{\headrulewidth}{0pt}
\renewcommand{\footrulewidth}{0pt}}
\pagestyle{custom}
%\end{comment}

%below creates a new circuit symbol for AC sources
\tikzset{
         ac source IEC graphic/.style=
          {
           transform shape,
           circuit symbol lines,
           circuit symbol size = width 3 height 3,
           shape=generic circle IEC,
           /pgf/generic circle IEC/before background=
            {
             \pgftransformresetnontranslations
             \pgfpathmoveto{\pgfpoint{-0.8\tikzcircuitssizeunit}{0\tikzcircuitssizeunit}}
             \pgfpathsine{\pgfpoint{0.4\tikzcircuitssizeunit}{0.4\tikzcircuitssizeunit}}
             \pgfpathcosine{\pgfpoint{0.4\tikzcircuitssizeunit}{-0.4\tikzcircuitssizeunit}}
             \pgfpathsine{\pgfpoint{0.4\tikzcircuitssizeunit}{-0.4\tikzcircuitssizeunit}}
             \pgfpathcosine{\pgfpoint{0.4\tikzcircuitssizeunit}{0.4\tikzcircuitssizeunit}}
             \pgfusepathqstroke
            }
          }
        }
% end of circuit symbol
%\begin{document}
%%%end individual beginning code/,$d


%  \begin{titlepage}
%    \vspace*{\fill}
%    \begin{center}
%      \huge{{\bf TITLE1}}\\[0.4cm]
%      \huge{TITLE2}\\[0.4cm]
%      \LARGE{Laboratory Manual}\\[0.4cm]
%      \large{SEASON YEAR}
%    \end{center}
%    \vspace*{\fill}
%  \end{titlepage}
%\maketitle

%\begin{spacing}{0.5}
%\tableofcontents
%\end{spacing}

%NEW PHYS 497 PACKAGES AND COMMANDS

%Subcaption package: Allows subfigures to be placed side by side, and labeled with individual captions (Added June 1, 2016)
\usepackage{subcaption}

%Array package: Allows for addiation specifications in arrays (Added May 6, 2016)
\usepackage{array}

%newcolumntype: Allows one to specify a fixed column width (Added May 6, 2016)
\newcolumntype{L}[1]{>{\raggedright\let\newline\\\arraybackslash\hspace{0pt}}m{#1}}
\newcolumntype{C}[1]{>{\centering\let\newline\\\arraybackslash\hspace{0pt}}m{#1}}
\newcolumntype{R}[1]{>{\raggedleft\let\newline\\\arraybackslash\hspace{0pt}}m{#1}}

%circuits.logic.US, circuits.logic.IEC: For drawing logic gates in Tikz (Added May 6, 2016) 
\usetikzlibrary{circuits.logic.US,circuits.logic.IEC}

\newcommand{\PGT}{ %PGT: positive going transition
\begin{tikzpicture}
\draw[-angle 60] (0,0) -- (0,5pt);
\draw (0,5pt) -- (0,6pt) -- (5pt,6pt);
\draw (-5pt,0) -- (0,0);
\end{tikzpicture}
}





%TEST
\usepackage{geometry}
\pagestyle{fancy}

%\usepackage[caption=false]{subfig}

%\makeatletter
%\renewenvironment{figure}[1][htbp]{%
%  \@tufte@orig@float{figure}[#1]%
%}{%
%  \@tufte@orig@endfloat
%}

%\renewenvironment{table}[1][htbp]{%
%  \@tufte@orig@float{table}[#1]%
%}{%
%  \@tufte@orig@endfloat
%}
%\makeatother

% use instead of subfigure
\makeatletter
\newenvironment{multifigure}[1][htbp]{%
  \@tufte@orig@float{figure}[#1]%
}{%
  \@tufte@orig@endfloat
}
\makeatother

\makeatletter
\newenvironment{mainfigure}[1][htbp]{%
\@tufte@orig@float{figure}[#1]
\begin{adjustwidth}{}{-153pt}}
{\end{adjustwidth}\@tufte@orig@endfloat}%
\makeatother

\makeatletter
\newenvironment{maintable}[1][htbp]{%
\@tufte@orig@float{table}[#1]
\begin{adjustwidth}{}{-153pt}}
{\end{adjustwidth}\@tufte@orig@endfloat}%
\makeatother

%%%% Labatorial Cross-over labs need this code. This should be temporary PG Dec 7, 2016

\newcounter{questioncounter}
\setcounter{questioncounter}{0}
\newcounter{checkpointcounter}
\setcounter{checkpointcounter}{0}
\newcounter{figurecounter}
\setcounter{figurecounter}{0}
%%%%%%%%%%%%%%%%%%%%%%%%%%%%%%%%%%%%%%%%%%%%%%%%%%%%%%%

\newcommand{\checkpoint}{
 \fbox{\begin{minipage}{0.2\textwidth}
 %\includegraphics[width=0.5\textwidth]{stop}
 \end{minipage}
 \begin{minipage}{1.0\textwidth}
 {\bf CHECKPOINT \addtocounter{checkpointcounter}{1} \arabic{checkpointcounter}: Before moving on to the next part, have your TA check the results you obtained so far.}
 \end{minipage}}}

%%% end labatorial cross-over code.

% New environment for placing figure captions under the figure
%\makeatletter
%\newenvironment{mainfigure}{\textwidth}[1][htbp]{%
%\@tufte@orig@float{figure}[#1]%
%}{%
%\@tufte@orig@endfloat
%}
%\makeatother

\begin{document}


\chapter{\bf Educational Laboratory Website Proposal}
\vspace{-1.5cm}
\today
\begin{adjustwidth}{}{-153pt}

\section{\bf Summary}
The goal of the educational laboratory website is to give everyone in the department access to the educational lab. It is being designed as a repository of knowledge that will enable faculty and staff to better share resources, to collaborate on development projects, and safeguard against single points of failure. Ultimately it is about improving the quality of experiments, increasing the professionalism of the labs, and protecting our intellectual resources. \\


\section{\bf Highlights of functions and features}

\begin{itemize}
\item Easy access to all available experiment documents both past and present.
\item Ability to filter labs by a range of parameters.
\item Easy distribution of editable documents to authorized personnel.
\item Professional look to the education lab's on-line presence.
\item Platform for the transfer of all lab related knowledge.
\item Access to support document.
\item Access to lab document templates.
\item Complete equipment database, including status and location.
\item Mobile friendly.
\end{itemize}


\section{\bf Resources needed}

The next step is to turn the current prototype in its current state into a final product that will be supported by Sci-IT. Much of the remaining work can be completed by PJL technicians or by Sci-IT. However, in talking to the head of Sci-IT it was determined that the most effective way of making this transition would be to work with the original programmer to finish the project and hand it over to Sci-IT for hosting and maintenance.\\ 

The bottom line is that to finish the project we will need to procure 150 hours of time from the original programmer (OP). 


\section{\bf What we have now}
As of today there is a working prototype, which can be viewed by going to pjl.ucaglary.ca and clicking on "Future of the PJL". There are 182 labs in our repository, with versions going back as much as seven iterations, for a total of 744 different documents. \\

Filtering tools are working, which make it easy for the user to quickly pair down from 744 documents to find only labs they want. They could been labs used a particular course, or in a year, semester etc. It is also possible to quickly find labs that haven't been done for years, and have been forgotten. \\

The ability to search labs by keywords. Several of the labs have had keywords assigned to them that identify several characteristics of the lab, from the branch of physics they are a part of, to what physics laws they address, or what mathematical tools are required in order to be able to complete the lab.\\

Further information, including access to all code and organizational diagrams, can be found at\\ github.com/pgimby/pjl-web.


\section{\bf What we need to finish}

\begin{itemize}
\item Finish developing the remaining features of the website. (150 hrs - OP)
\item Authentication method for downloading source code. (Sci-IT)
\item Security audit. (Sci-IT)
\item Finish building the equipment database. (70 hrs - Tech)
\item Finish adding keywords to the repository database. (35 hrs - Tech and/or Instructor)
\item Build database of equipment manuals. (100 hrs - Summer Student)
\item Transition final product to final destination as determined by Sci-IT. (Sci-IT, Tech, OP)
\end{itemize}


\section{\bf Benefit to the department}

The main advantage of the website and databases to the department as a whole is that it gathers and displays the collective knowledge of staff, both past and present, in a way that is accessible and user friendly. It provides a platform for collaborating on development projects. Finally, it enables the smooth transfer of lab knowledge, and removes single points of failure. Vital information will no longer be stored in various places, and known about by only one person.

\section{\bf Benefit to lab developers}
The document repository has been reconstructed to allow for better collaboration for lab development projects. From the site an instructor will be able to download the latest version of a lab to make changes. They will also have access to any data that has been collected, and will be able to look into the history of the lab, to see how it has been used in the past. They will be able to view manuals for the equipment they want to use, and even be able to tell how many units we have of any type of equipment they want. Someone working to develop a lab will also be able to access papers that were used in the original development.

\section{\bf Benefit to technical staff}
On the most basic level this website will reduce the complexity of setting up labs from a job that requires significant training, to a job that could be completed with little to no training. The mobile version of the website will provide an equipment list for each lab, a location for each piece of equipment, and maintenance status for each type of equipment. If the person setting up the lab is new, they can simply open the document on their mobile device and view a photograph of the setup. 


\end{adjustwidth}
\end{document}
