%Created on:  	Feb 5, 2016  	Created by: P. Gimby
%Reviewed on: 	Feb 10, 2016   	Reviewed by: P., M. Wieser
%Edited on:   	Feb 16, 2016	Edited by: P. Gimby                Changed name, added process for conflict resolution
%Reviewed on: 	Feb 17, 2016   	Reviewed by: P. Gimby, M. Wieser   Added "Lab Document" definition, corrected spelling and grammer
%Edited on:   	Feb 24, 2016	Edited by: P. Gimby
%Reviewd on:  	Mar 14, 2016	Reviewed by: UAC
%Edited on:	  	Mar	17, 2106	Edited by: P. Gimby 				Included recommendations from UAC
%Review on:		Apr 8, 2016		Reviewed by: UAC
%Edited on:		Apr 8, 2016		Edited by: P. Gimby					Included recommendations from UAC
%Edited on:		Apr 20, 2016	Edited by: P. Gimby					Removed sections with regards to specifics about edits, and updated the edit cycle
%Reviewed on:	Apr 20, 2016	Reviewed by: Jo-Anne Brown			Suggested changes to general comments, and 
%Reviewed on: 	Apr 22, 2016 	Reviewed by: UAC
%Edited on:		Apr 27, 2016	Edited by : P. Gimby				Fixed Typos
%Reviewed on: 	Sep 22, 2016 	Reviewed by: UAC		
%Edited on:		Sep 23, 2016	Edited by : P. Gimby				Made changes per the UAC discussion
%Edited on:		Sep 26, 2016	Edited by : P. Gimby				Made changes per discussion with UAC chair
%Edited on:		Sep 29, 2016	Edited by : P. Gimby				Made changes per discussion with UAC
%Approved on:                   Approved by:

% Beginning code for all standard physics latex documents

%Created on: May 8, 2014    Edited by: Wesley Kyle
%Edited on:	May 12, 2016	Edited by: P. Gimby - cleaned up the code to remove unneeded packages
%Edited on:	May 13, 2016	Edited by: P. Gimby - collected a few more packages used in 325.
%Edited on:	May 16, 2016	Edited by: P. Gimby - fixed page numbering error.
%Edited on: May 20, 2016	Edited by: Alex Shook - Added packages for 497

\documentclass[justified]{tufte-book}
\usepackage{graphicx} % allow embedded images
\setkeys{Gin}{width=\linewidth,totalheight=\textheight,keepaspectratio}
\usepackage{amsmath}  % extended mathematics
\usepackage{bm}  % bold font in math mode
\usepackage{longtable} %lets long tables flow into multiple pages instead of running off the page or having to break tables up manually
\usepackage{booktabs} % book-quality tables
\usepackage{units}    % non-stacked fractions and better unit spacing
\usepackage{multicol} % multiple column layout facilities
\usepackage{tikz} %for drawing nice pictures
\usepackage{indentfirst} % makes first line of each new section be indented
\usepackage{enumitem} % extended options for the enumerate environment
\usepackage{soul} % gives more typestting options like spacing, underline, and strike-through
\usepackage{marvosym} %extra symbols package
\usepackage{multirow} % for special table controls
\usepackage[singlelinecheck=false]{caption} % allow captions w/o figure number
\captionsetup{compatibility=false} % corrects in issue with the caption package
\usepackage{float} % allows for contorl over position of figures and tables
\allowdisplaybreaks % allows equations to span two pages if needed
\usepackage{mathrsfs} % fancy math symbols
\usepackage{multirow} % for special table controls
\usetikzlibrary{arrows,shapes,snakes,calc,patterns,3d} % addon to tikz
\usetikzlibrary{circuits.ee.IEC} % addon to tikz
\usepackage{pgfplots} % package for making plots of functions
\usepackage{gensymb} % symbols i,e. degrees
\usetikzlibrary{decorations.pathmorphing} % to draw the springs
\tikzset{circuit declare symbol = ac source}
\tikzset{set ac source graphic = ac source IEC graphic}
\usepackage{changepage} % allows for full page environment
\usepackage{comment} % allows comment tags for large sections

% define new page style that puts page numbers in the middle
%\begin{comment}
\fancypagestyle{custom}{
\fancyhf{} % clear all header and footer fields
\fancyheadoffset{0pt}
\fancyfootoffset{0pt}
\fancyfoot[C]{\thepage}
\renewcommand{\headrulewidth}{0pt}
\renewcommand{\footrulewidth}{0pt}}
\pagestyle{custom}
%\end{comment}

%below creates a new circuit symbol for AC sources
\tikzset{
         ac source IEC graphic/.style=
          {
           transform shape,
           circuit symbol lines,
           circuit symbol size = width 3 height 3,
           shape=generic circle IEC,
           /pgf/generic circle IEC/before background=
            {
             \pgftransformresetnontranslations
             \pgfpathmoveto{\pgfpoint{-0.8\tikzcircuitssizeunit}{0\tikzcircuitssizeunit}}
             \pgfpathsine{\pgfpoint{0.4\tikzcircuitssizeunit}{0.4\tikzcircuitssizeunit}}
             \pgfpathcosine{\pgfpoint{0.4\tikzcircuitssizeunit}{-0.4\tikzcircuitssizeunit}}
             \pgfpathsine{\pgfpoint{0.4\tikzcircuitssizeunit}{-0.4\tikzcircuitssizeunit}}
             \pgfpathcosine{\pgfpoint{0.4\tikzcircuitssizeunit}{0.4\tikzcircuitssizeunit}}
             \pgfusepathqstroke
            }
          }
        }
% end of circuit symbol
%\begin{document}
%%%end individual beginning code/,$d


%  \begin{titlepage}
%    \vspace*{\fill}
%    \begin{center}
%      \huge{{\bf TITLE1}}\\[0.4cm]
%      \huge{TITLE2}\\[0.4cm]
%      \LARGE{Laboratory Manual}\\[0.4cm]
%      \large{SEASON YEAR}
%    \end{center}
%    \vspace*{\fill}
%  \end{titlepage}
%\maketitle

%\begin{spacing}{0.5}
%\tableofcontents
%\end{spacing}

%NEW PHYS 497 PACKAGES AND COMMANDS

%Subcaption package: Allows subfigures to be placed side by side, and labeled with individual captions (Added June 1, 2016)
\usepackage{subcaption}

%Array package: Allows for addiation specifications in arrays (Added May 6, 2016)
\usepackage{array}

%newcolumntype: Allows one to specify a fixed column width (Added May 6, 2016)
\newcolumntype{L}[1]{>{\raggedright\let\newline\\\arraybackslash\hspace{0pt}}m{#1}}
\newcolumntype{C}[1]{>{\centering\let\newline\\\arraybackslash\hspace{0pt}}m{#1}}
\newcolumntype{R}[1]{>{\raggedleft\let\newline\\\arraybackslash\hspace{0pt}}m{#1}}

%circuits.logic.US, circuits.logic.IEC: For drawing logic gates in Tikz (Added May 6, 2016) 
\usetikzlibrary{circuits.logic.US,circuits.logic.IEC}

\newcommand{\PGT}{ %PGT: positive going transition
\begin{tikzpicture}
\draw[-angle 60] (0,0) -- (0,5pt);
\draw (0,5pt) -- (0,6pt) -- (5pt,6pt);
\draw (-5pt,0) -- (0,0);
\end{tikzpicture}
}





%TEST
\usepackage{geometry}
\pagestyle{fancy}

%\usepackage[caption=false]{subfig}

%\makeatletter
%\renewenvironment{figure}[1][htbp]{%
%  \@tufte@orig@float{figure}[#1]%
%}{%
%  \@tufte@orig@endfloat
%}

%\renewenvironment{table}[1][htbp]{%
%  \@tufte@orig@float{table}[#1]%
%}{%
%  \@tufte@orig@endfloat
%}
%\makeatother

% use instead of subfigure
\makeatletter
\newenvironment{multifigure}[1][htbp]{%
  \@tufte@orig@float{figure}[#1]%
}{%
  \@tufte@orig@endfloat
}
\makeatother

\makeatletter
\newenvironment{mainfigure}[1][htbp]{%
\@tufte@orig@float{figure}[#1]
\begin{adjustwidth}{}{-153pt}}
{\end{adjustwidth}\@tufte@orig@endfloat}%
\makeatother

\makeatletter
\newenvironment{maintable}[1][htbp]{%
\@tufte@orig@float{table}[#1]
\begin{adjustwidth}{}{-153pt}}
{\end{adjustwidth}\@tufte@orig@endfloat}%
\makeatother

%%%% Labatorial Cross-over labs need this code. This should be temporary PG Dec 7, 2016

\newcounter{questioncounter}
\setcounter{questioncounter}{0}
\newcounter{checkpointcounter}
\setcounter{checkpointcounter}{0}
\newcounter{figurecounter}
\setcounter{figurecounter}{0}
%%%%%%%%%%%%%%%%%%%%%%%%%%%%%%%%%%%%%%%%%%%%%%%%%%%%%%%

\newcommand{\checkpoint}{
 \fbox{\begin{minipage}{0.2\textwidth}
 %\includegraphics[width=0.5\textwidth]{stop}
 \end{minipage}
 \begin{minipage}{1.0\textwidth}
 {\bf CHECKPOINT \addtocounter{checkpointcounter}{1} \arabic{checkpointcounter}: Before moving on to the next part, have your TA check the results you obtained so far.}
 \end{minipage}}}

%%% end labatorial cross-over code.

% New environment for placing figure captions under the figure
%\makeatletter
%\newenvironment{mainfigure}{\textwidth}[1][htbp]{%
%\@tufte@orig@float{figure}[#1]%
%}{%
%\@tufte@orig@endfloat
%}
%\makeatother

%\usepackage{endnotes}
%\let\footnote=\endnote
\begin{document}

\chapter{Teaching Lab Development Policies and Procedures}

\begin{adjustwidth}{}{-153pt}
\nointerlineskip\leavevmode

\vspace{-1cm}

\noindent Revised \today

\subsection{\bf General Comments}
Several members of department are involved in the development of the teaching laboratory curriculum. This policy is intended to provide a framework that will allow for greater collaboration amongst developers. It is meant to be a mechanism for the execution of a lab development plan as outlined by the teaching laboratory development committee. This policy outlines a quality control process to ensure that the labs presented to the students align with the departments philosophies, educational goals, and safety requirements, as well as ensure the integrity of vital laboratory documents. 


\subsection{\bf Teaching Support Office - Contact Information}
%first column
\begin{minipage}{0.5\textwidth}
{\bf Peter Gimby}\\
Junior Labs Supervisor\\
Office: Science Theaters $068$\\
Phone: $403.222.5403$\\
Email: pgimby@ucalgary.ca\\
Website: pjl.ucalgary.ca\\
\end{minipage}
%second column
\begin{minipage}{0.5\textwidth}
{\bf Wesley Ernst}\\
Undergrad Support Technician\\
Office: Science Theaters $068$\\
Phone: $403.220.7401$\\
Email: wernst@ucalgary.ca\\
Website: pjl.ucalgary.ca\\
\end{minipage}

\section{\bf Purpose for Policy}
%\vspace{-0.25cm}
\begin{enumerate}[noitemsep]
\item Ensure version control of lab documents.
\item Define process for editing lab documents.
\item Identify timeline for the completion of edits to lab documents.
\item Provide quality control for edits to lab experiments.
\item Define mechanism for the distribution of lab documents.
\end{enumerate}

\newpage

\section{\bf Abbreviations} 
\begin{itemize}[noitemsep]
\item AHU - Associate Head, Undergraduate%{\color{red} \footnote{Is this the correct title for David F?}}
\item JLS - Junior Laboratory Supervisor
\item TLDC -	Teaching Lab Development Committee
\item TSO - Teaching Support Office
\item UAC - Undergraduate Affairs Committee
\item ULC - Undergraduate Learning Coordinator
\item UST - Undergraduate Support Technician
\end{itemize}

\section{\bf Cycle for Making Edits}
\begin{enumerate}[noitemsep]
\item Edits to lab documents are commissioned by the UAC, and overseen by the TLDC.
\item Person(s) is (are) assigned to make the edits (the "editor": instructor, ULC, AHU, JLS, UST, summer student, etc.)
\item Source code is "checked out" of the official document repository.
\item Edits are made to documents.
\item Edits are reviewed by Instructor, JSL, and AHU, or delegates. 
\begin{itemize}
\item {\bf Due 8 am 15 working days before the start of semester}.
\end{itemize}
\item Revisions are made by the editor in response to comments from review process.
\item Repeat 4 and 5 as needed.
\item Final version is submitted to AHU for approval.
\begin{itemize}
\item {\bf Due 8 am 5 working days before the start of semester}.
\end{itemize}
\item Edits are made by the TSO, in conjunction with the editor, in response to concerns raised by AHU.
\item Source code is "checked back in" and returned to the official repository located in the TSO.
\begin{itemize}
\item {\bf Due by 12 pm on the last working day before the start of semester}.
\end{itemize}
\end{enumerate}

Please note that while some steps in the review process have due dates attached to them, it is highly recommended that documents be submitted for review as soon as possible. This will help to reduce the number of documents needing to be reviewed during the three weeks leading up to the start of the semester.

\subsection{\bf Definition of Lab Documents}
\begin{enumerate}[noitemsep]
\item Experiment .tex document (source code for student exercises and companion guide)
\item Compiled pdf files (pdfs of student exercise and companion guide)
\item External figures that are embedded into .tex file (pictures or complex figures, as well as source code)
\item Simulations (jar, html, python, etc.)
\item Experimental data (sample graphs, data, equipment calibrations, etc) 
\end{enumerate}

\newpage
\section{\bf Distribution and Archiving  of Lab Documents}
\noindent Distribution and Archiving of Lab Documents is to take place only after the documents have been approved by the AHU, and the source code had been returned to the official repository.

\begin{enumerate}[noitemsep]
\item Official versions of the documents will be posted on the TSO website (pjl.ucalgary.ca), and a "snap-shot" of the documents will be placed in the main office shared drive.
\item The instructor can download copies of the experiment manuals from the TSO website, or the department shared drive, and distribute them to students and TAs (for example, via D2L), or by directing their students to the TSO website.
\item All files needed for the execution of the lab experiment will be added to the lab computers by the TSO. This includes any applets as well as pdfs of the lab exercise.
\item Pdfs of all lab exercises will be added to the staff resources section of the TSO website.
\item All files will be added to the archives. The archives exist as both a hard copy and electronic copy of the official lab document used for each semester. The archived documents are never edited. PDFs of all archived documents will be posted on the TSO website.  
\end{enumerate}
\vspace {2cm}

\noindent\begin{tabular}{ll}
\makebox[3.5in]{\hrulefill} & \makebox[1.5in]{\hrulefill}\\
PHAS - Associate Head, Undergraduate & Date\\[8ex]% adds space between the two sets of signatures
\makebox[3.5in]{\hrulefill} & \makebox[1.5in]{\hrulefill}\\
PHAS - Department Head & Date\\
\end{tabular}

\end{adjustwidth}
\end{document}
