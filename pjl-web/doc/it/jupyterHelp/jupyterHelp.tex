% Version Control Log. Please make an entry bellow when editing.
%
% Created  on:  Oct 7, 2019        	Created  by:  Peter Gimby



% Jupyter instructions%----------------------------------------------------------------
% Document initialization
%----------------------------------------------------------------

\documentclass{../../../assets/LabArx-Dev} 	% PJL lab class

\begin{document}

\Logo{../../../assets/logo.jpg}
\Version{1.2}
%----------------------------------------------------------------
% start document body - DO NOT REMOVE THIS LINE
%----------------------------------------------------------------

%----------------------------------------------------------------
% Title Page and Experiment Information 
%----------------------------------------------------------------

\TitleVar{Running Jupyter on Kubuntu 18}{24}{30}
\maketitle
\fancyfoot{}
%\turnOffFooter{nofoot}
%----------------------------------------------------------------
% Main Body
%----------------------------------------------------------------
 
\begin{enumerate}

\item {\bf Starting Jupyter}
	
	Open up a command console (ie, konsole or yakuake) and run one 		of the following commands.
	\begin {itemize}
		\item jupyter-notebook
		\item jupyter notebook
		\item jupyter-lab
		\item jupyter lab
	\end{itemize}
	Alternatively, you can click on the background and type any of the above commands. This will open jupyter from an application search bar.  

\item{\bf Installing python modules.}
	
	All modules must be installed by the system administrator. To 		request a new package be added email phas.edulabs@ucalgary.ca 		with as much the following information as possible.

	\begin{itemize}
		\item{Name of package.}
		\item{Very basic example of code you want to run with the 			module.}
	\end{itemize}
	
	We will get the package installed as soon as possible.
	
\item {\bf Note on vpython.}

	A vpython notebook must be complied in google chrome.\\If the 		notebook opens in firefox.
	
	\begin{itemize}
		\item In the output in the command console used to start 			jupyter there will be a line that looks like.\\

\begin{lstlisting}[backgroundcolor = \color{light-gray}]
Or copy and paste one of these URLs:
    http://localhost:8889/ token=cb14eee9a5226cec0c9748245beec636cc2315ca2444922c
	\end{lstlisting}

		\item Copy the url from the console into google chrome.
		\item It will as for a password which will be everything 			after "token=" in the url.
		\item For this example the password is

\begin{lstlisting}[backgroundcolor = \color{light-gray}]
 cb14eee9a5226cec0c9748245beec636cc2315ca2444922c
\end{lstlisting}
	\end{itemize}

\end{enumerate}

\end{document}
