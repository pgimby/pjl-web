%Created on:  	Jun 15, 2016  	Created by: P. Gimby


\documentclass[justified]{tufte-book}
%\usepackage{graphicx} % allow embedded images
%\usepackage{fancyvrb} % extended verbatim environments
%\setkeys{Gin}{width=\linewidth,totalheight=\textheight,keepaspectratio}
%\usepackage{amsmath}  % extended mathematics
%\usepackage{booktabs} % book-quality tables
%\usepackage{units}    % non-stacked fractions and better unit spacing
%\usepackage{multicol} % multiple column layout facilities
%\fvset{fontsize=\normalsize}% default font size for fancy-verbatim environments
\usepackage{tikz} %for drawing nice pictures
\usepackage{pgfplots}
\usepackage{indentfirst}
\usepackage{enumitem}
\usepackage{verbatim}
\usepackage{soul}
\usepackage{lastpage}
%\usepackage[normalem]{ulem}

\fancypagestyle{fullpage}{%
\fancyhf{} % clear all header and footer fields

\fancyheadoffset{0pt}
\fancyfootoffset{0pt}
\fancyfoot[R]{\thepage \ of  \pageref{LastPage}}
\fancyfoot[L]{\itshape\nouppercase\leftmark}
\renewcommand{\headrulewidth}{0pt}
\renewcommand{\footrulewidth}{0pt}}
\usepackage{latexsym}

\begin{document}
\def\Item{\item[$\Box$]}
\newgeometry{margin=1in}
\pagestyle{fullpage}

\chapter{Annual Laboratory Safety Procedures}\label{chp:ciLabInspections}
%\fancyfoot[R]{\ref{chp:puldhp}}
%\rhead{Physics Undergraduate Lab Document Handling Policy}
%%%start document%%% DO NOT REMOVE THIS LINE
\vspace{-1cm}
%\minitoc
\noindent Revised \today %June 20$^{th}$, 2016

\subsection{\bf General Comments}
This document covers instructions on how to take care of lab safety items that need to be completed once a year. The activities outlined in this document rely heavily on chematix chemical management system https://ucalgary.chematix.com/Chematix/

\section{\bf Processes Covered by Document}
\begin{enumerate}
	\item Hazardous Waste Disposal.
	\item Reconciling Chemical Inventory.
	\item Hazard Assessment Control Form (HACF) Review.
	\item Laboratory Self Inspections.
\end{enumerate}

\section{\bf Hazardous Waste Disposal}
\subsection{Requirements}
\begin{itemize}
	\Item Store liquid waste stored in $1000$ mL Polypropylene bottle (ie VWR PT\# 				$414004-183$). 
	\Item Liquid waste marked with contents and activity if radioactive.
	\begin{itemize}
		\item For radium experiment bottles should be labeled with "5.476 Bq/mL $^{226}				$Radium 	Salt".
	\end{itemize}
	\Item Solid waste stored in zip-lock bags.
	\Item Solid waste containers labeled with contents including any contaminants.
	\begin{itemize}
		\item For radium experiment bags this would be "Debris Contaminated with $^{226}				$Radium	Salt".
	\end{itemize}
\end{itemize}


\subsection{Procedure}
\begin{enumerate}
	\item Enter Chematix.
	\item Click on "Waste" under the main header
	\item Click on "Create Waste Pickup Worksheet"
	\item Select location for pickup from drop down menu.
	\item Add instructions in comment box indicating how HAZMAT can gain access to room for 		pickup.
	\item Create chemical waste card(s).
	\begin{enumerate}
		\item Click on "Add more waste"
		\item Select "Contaminated Materials Waste Card"
		\item Indicate location on drop down menu
		\item Check "Radioactive" under "Contamination Type"
		\item Report quantity, size, and contaminate of containers. Containers with identical 		contaminates can be entered on the same waste card.
		\item Print and sign waste card and leave it with disposal. HAZMAT will need this.
	\end{enumerate}
	\item Once all waste cards are added click "Submit Waste for Pickup" 
\end{enumerate}


\section{\bf Reconciling Chemical Inventory}
\subsection{Requirements}
\begin{itemize}
	\Item Every chemical has a bar code, or is on the undeclared list.
	\Item Every chemical's amount remaining is up to date.
	\Item Every chemical is in it's designated storage unit.
	\Item Every storage unit has a bar code.
	\Item Every chemical that is not needed is disposed of.
	\Item All chemical waste has been removed by HAZMAT.
	\Item Chemical inventory has been reconciled through Chematix.
\end{itemize}

\subsection{Procedure}
\begin{enumerate}
	\item Enter Chematix.
	\item Update amounts remaining of  all chemical.
	\begin{enumerate}
		\item Click "Inventory" under main header.
		\item Click "Manage My Inventory" under "Manage Lab Inventory"
		\item Click on "Toggle" button to select all labs, and click "Search Active Lab 				Inventory"
		\item Click on "Barcode" of each line item and confirm that the "Content Size" is 			accurate.
		\item Adjust "Content Size"
		\begin{enumerate}
			\item From main inventory page click "Adjust Container Quantity" under "Add Items 			to Inventory"
			\item Search barcode of container to adjust
			\item Enter amount removed from original content size into the box labeled 					"Removed	 Content Size/Units
			\item Select the correct units and "Commit Modification"
		\end{enumerate}
	\end{enumerate}
	\item Obtain a bar code scatter from PHAS main office
	\item Follow scanner guide for scanning chemicals. Remember to scan storage container and 	then individual containers.
	\item Input barcodes
	\item RECONCILE
	\begin{enumerate}
		\item Click "Inventory" under main header.
		\item Click "Reconcile Multiple Laboratory Inventories" under "Inventory Reconciliation"
		\item Select all rooms to reconcile.
		\item Click on "Reconcile Selected Laboratories"
		\item Adjust "Content Size"
		\item Click 
	\end{enumerate}
\end{enumerate}



\newpage
\section{\bf HACF Review}
\subsection{Requirements}
\begin{itemize}
	\Item Review Hazard Assessment Control Form
	\begin{itemize}
		\Item Staff HACF
		\Item Standard TA HACF
		\Item Advanced TA HACF
	\end{itemize}
	\Item Update HACFs in Safety Binders
	\Item Submit updated HACFs to EH\&S
\end{itemize}

\section{\bf Laboratory Self Inspection}
\subsection{Requirements}
\begin{itemize}
	\Item Lab Inspections for $09,\ 17,\ 29,\ 39,\ 48/50$ entered in chematix
	\Item Administrative inspections for ST $25,\ 26,\ 30,\ 32,\ 34,\ 36,\ 37,\ 38,\ 42,\ 68$
\end{itemize}
\end{document}
